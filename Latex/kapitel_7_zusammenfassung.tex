\chapter{Zusammenfassung}
In dieser Arbeit wurde vorgestellt, welche Vorteile die parallele Datenverarbeitung erreichen kann. Es wurde aufgezeigt, welche Voraussetzungen erfüllt sein müssen und wie sich die Parallelisierung umsetzen lässt und die Entwicklung von Multithread Anwendungen, wurde an zwei Beispielprogrammen und zugehörigen Bibliotheken demonstriert.\\
Dabei wurde klar, dass die hardwareseitigen Voraussetzungen bei aktuellen Systemen in der Regel zu genüge erfüllt sind. Außerdem wurde gezeigt, dass die softwareseitigen Schnittstellen immer besser und effizienter nutzbar werden. Dadurch nimmt der zusätzliche Aufwand sowohl bei der Entwicklung als auch bei der Nutzung von Multithread Anwendungen immer weiter ab, sodass die Nutzung der parallelen Verarbeitung für viele Softwaresysteme immer effektiver und somit immer wichtiger wird. Teilweise eröffnet sie sogar ganz neue Dimensionen in Umfang aber auch Art der bearbeiteten Probleme.\\
Angesichts dessen sollte die Betrachtung der Parallelisierung in Zukunft einen höheren Stellenwert in der der Lehre, besonders im Informatikunterricht allgemeinbildender Schulen, in dessen Lehrplänen die Thematik derzeit keinerlei Berücksichtigung findet, erlangen. \cite{lehrplan_informatik}
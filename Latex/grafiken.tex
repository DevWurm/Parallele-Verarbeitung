\chapter*{Grafiken}

\begin{figure}[h]
\begin{center}
\includegraphics[scale=0.77]{img/komplexitat.pdf}
\caption{Qualitative Darstellung der durch die Zeitkomplexität zu erwartenden Laufzeit bei steigender Listenlänge $n$; $p=4$; $c=1$}
\label{komplexitat}
\end{center}
\end{figure}

\begin{figure}[h]
\begin{center}
\includegraphics[scale=0.25]{img/sequence_master_slave.pdf}
\caption{Sequenzdiagramm einer einfachen parallelisierten Anwendung nach dem Master-Slave-Prinzip}
\label{sequence_master_slave}
\end{center}
\end{figure}

\begin{figure}[h]
\begin{center}
\includegraphics[scale=0.5]{img/top_h.png}
\caption{Beispielausgabe eines Prozessmonitors bei einer parallelen Anwendungen (roter Rahmen)}
\label{top_h}
\end{center}
\end{figure}

\begin{figure}[h]
\begin{center}
\includegraphics[scale=0.76]{img/vergleich.pdf}
\caption{Laufzeitvergleich zwischen sequentiellem (rot) und parallelem (grün) Mergesortalgorithmus bei steigender Listenlänge}
\label{laufzeit}
\end{center}
\end{figure}

\begin{figure}[h]
\begin{center}
\includegraphics[scale=0.25]{img/compare.png}
\caption{Visueller Laufzeitvergleich zwischen Mergesort (Algoritmus 1) und Bubblesort (Algorithmus 2) mit merge\_sort\_bubble\_sort}
\label{compare}
\end{center}
\end{figure}
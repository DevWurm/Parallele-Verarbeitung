\chapter{Einleitung}
Das Prinzip der parallelen Verarbeitung ist ein grundlegender Baustein der modernen Informatik und Anwendungsentwicklung. Bei einer ständigen Steigerung der Anforderung an Zeiteffizienz, Reaktionsgeschwindigkeit und Verfügbarkeit von Computersystemen ist eine sequenzielle Ausführung von Softwarekomponenten längst nicht mehr ausreichend. Die Fähigkeit aktueller Hardwarekomponenten und -schnittstellen, mehrere Routinen simultan auszuführen, liefert die Möglichkeit diese Anforderungen zu erfüllen.

\section{Anwendungsfälle paralleler Verarbeitung}
Der wahrscheinlich bekannteste Anwendungsfall paralleler Verarbeitung ist die Parallelisierung von Problemlösungen zur Effektivitätssteigerung von Programmen, wie es etwa bei Kompressions-, Sortier- oder Analyseverfahren der Fall ist. Sie wird aber auch bei anderen Prozessen verwendet, bei denen mehrere (oft verschiedene) Routinen gleichzeitig ablaufen. Sei es bei der Entwicklung von Programmen mit \acp{GUI}, bei denen gleichzeitig Daten bearbeitet werden und die Oberfläche aktualisiert wird, bei der Echtzeiterfassung von Daten oder bei der Visualisierung von Algorithmeneffizienzen in der Didaktik.\\
Auch diese Beispiele sind nur ein Ausschnitt aus dem weiten Anwendungsfeld der parallelisierten Datenverarbeitung.

\section{Zielstellung}
Ziel dieser Arbeit ist es, zum einen die prinzipiellen Konzepte und Voraussetzungen sowie die Vorteile und die wachsende Notwendigkeit paralleler Verarbeitung darzustellen und zum anderen deren praktische Umsetzung und Auswirkung exemplarisch anhand von Beispielbibliotheken und -anwendungen zu demonstrieren.

\section{Hinweise}
Alle Grafiken und Tabellen befinden sich im Anhang, um die Übersicht zu verbessern. Genaueres dazu ist dem Tabellen und Abbildungsverzeichnis zu entnehmen. In dieser Arbeit sind nur die nötigsten Quellcode Abschnitte abgedruckt. Der gesamte Quellcode, alle Messdaten sowie eine Kopie dieser Arbeit befinden sich auf dem beiliegenden Datenträger.

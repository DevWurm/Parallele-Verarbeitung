\chapter*{Tabellen}

\begin{table}[h]
\begin{center}
\begin{tabular}{|c|l|}
\hline
\textbf{Kriterium} & \textbf{Wert}\\
\hline
Betriebssystem & Ubuntu 14.04LTS 64bit\\
\hline
Gerät & MSI GE60 2PE Apache Pro\\
\hline
Mainboard & MS-16GF (REV:0.B)\\
\hline
Prozessor & Intel(R) Core(TM) i7-4710HQ\\
Kerne & 4\\
Threads (SMT) & 8\\
Taktfrequenz & 2.5GHz - 3.5GHz\\
Architektur & 64bit\\
\hline
\ac{RAM} & Fujitsu AM1L16BC8R2-B1QS\\
Speicher & 8GB\\
Art & DDR3 (1600MHz)\\
\hline
C++ Compiler & g++\\
Version & 4:4.8.2-1ubu\\
\hline
\end{tabular}
\caption{Technische Daten des Testsystems}
\label{testsystem}
\end{center}
\end{table}

\begin{table}[h]
\begin{tabular}{|p{5cm}|p{5cm}|p{4cm}|}
\hline
\textbf{Argument} & \textbf{Bedeutung} & \textbf{Argumente}\\
\hline
\texttt{-t; - -type} & Datentyp der Liste(n) (Bezeichner entsprechen den C++ Datentypen)& int, short, long, double, float\\
\hline
\texttt{-d; - -display} (optional) & Prozessdaten (Listenlänge, Sortierbegin-/ende, Sortierdauer) ausgeben (besteht die Datei werden die Daten angehangen)& keine\\
\hline
\texttt{-p; - -parallel} (optional) & Parallelen Sortieralgorithmus verwenden& keine\\
\hline
\texttt{-O; \mbox{- -output-information}} (optional) & Prozessdaten (Listenlänge, Sortierdauer) als \ac{CSV} Datei ausgeben& Dateipfad\\
\hline
\texttt{-i; \mbox{- -input}} & Als Datenquelle \ac{CSV} Datei verwenden (entweder -r oder -i muss verwendet werden)& Dateipfad\\
\hline
\texttt{-o; \mbox{- -output}} (optional) & Sortierte Daten als \ac{CSV} Datei ausgeben (besteht die Datei werden die Daten angehangen)& Dateipfad\\
\hline
\texttt{-r; \mbox{- -random}} & Zufällig erzeugte Daten als Quelle verwenden (entweder -r oder -i muss verwendet werden)& Listenlänge (Ganzzahl, Standard: 1)\\
&& Geringster Wert (Typ der Liste, Standard: 0)\\&&\\
&& Höchster Wert (Typ der Liste, Standard: 1)\\
\hline
\end{tabular}
\caption{Kommandozeilenargumente des Programme merge\_sort\_list}
\label{arguments_merge_sort_list}
\end{table}



\begin{table}[h]
\begin{tabular}{|p{5cm}|p{5cm}|p{4cm}|}
\hline
\textbf{Argument} & \textbf{Bedeutung} & \textbf{Argumente}\\
\hline
\texttt{-t; - -type} & Datentyp der Liste(n) (Bezeichner entsprechen den C++ Datentypen)& int, short, long, double, float\\
\hline
\texttt{-d; - -display} (optional) & Prozessdaten (Listenlänge, Sortierbegin-/ende, Sortierdauer) ausgeben (besteht die Datei werden die Daten angehangen)& keine\\
\hline
\texttt{-O; \mbox{- -output-information}} (optional) & Prozessdaten (Listenlänge, Sortierdauer) als \ac{CSV} Datei ausgeben& Dateipfad\\
\hline
\texttt{-i; \mbox{- -input}} & Als Datenquelle \ac{CSV} Datei verwenden (entweder -r oder -i muss verwendet werden)& Dateipfad\\
\hline
\texttt{-o; \mbox{- -output}} (optional) & Sortierte Daten als \ac{CSV} Datei ausgeben (besteht die Datei werden die Daten angehangen)& Dateipfad\\
\hline
\texttt{-r; \mbox{- -random}} & Zufällig erzeugte Daten als Quelle verwenden (entweder -r oder -i muss verwendet werden)& Listenlänge (Ganzzahl, Standard: 1)\\
&& Geringster Wert (Typ der Liste, Standard: 0)\\&&\\
&& Höchster Wert (Typ der Liste, Standard: 1)\\
\hline
\end{tabular}
\caption{Kommandozeilenargumente des Programme merge\_sort\_bubble\_sort}
\label{arguments_compare}
\end{table}